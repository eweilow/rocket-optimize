\documentclass[11pt]{article}
\usepackage[cm]{fullpage}
\usepackage{graphicx}
\usepackage{caption}
\usepackage{subcaption}
\usepackage[section]{placeins}
\usepackage{float}
\usepackage{amsmath}
\usepackage{multicol}

\setlength{\columnsep}{1cm}

\title{%
  Optimizing the ascent trajectory for an orbital class launch vehicle \\
  \Large Final project in SI1336\\
}
\author{Erik Weilow}


\newcommand{\triplefigure}[3]{
\begin{figure}[H]
  \centering
  \begin{minipage}{0.3\textwidth}
    \centering
    \includegraphics[width=\textwidth]{#1}
  \end{minipage}
  \begin{minipage}{0.3\textwidth}
    \centering
    \includegraphics[width=\textwidth]{#2}
  \end{minipage}
  \begin{minipage}{0.3\textwidth}
    \centering
    \includegraphics[width=\textwidth]{#3}
  \end{minipage}
\end{figure}
}
\newcommand{\doublefigure}[2]{
\begin{figure}[H]
  \centering
  \begin{minipage}{0.45\textwidth}
    \centering
    \includegraphics[width=\textwidth]{#1}
  \end{minipage}
  \begin{minipage}{0.45\textwidth}
    \centering
    \includegraphics[width=\textwidth]{#2}
  \end{minipage}
\end{figure}
}
\newcommand{\singlefigure}[1]{
\begin{figure}[H]
  \centering
  \begin{minipage}{0.4\textwidth}
    \centering
    \includegraphics[width=\textwidth]{#1}
  \end{minipage}
\end{figure}
}
\newcommand{\singlewiderfigure}[1]{
\begin{figure}[H]
  \centering
  \begin{minipage}{0.65\textwidth}
    \centering
    \includegraphics[width=\textwidth]{#1}
  \end{minipage}
\end{figure}
}

\let\oldabs\abs % Store original \abs as \oldabs
\let\abs\undefined % "Undefine" \abs
\DeclarePairedDelimiter\abs{\lvert}{\rvert}

\begin{document}
\maketitle
\newpage

\section{Parameters}

\section{The model}
To simulate the ascent of the rocket, a model of the physics involved is required.
In this model, it assumed that only main three forces are acting on the rocket: its thrust $T$, aerodynamic drag $D$ and gravity $G$.
All three of these forces need approximations in order to run the simulation with reasonable result.


\subsection{Coordinate system}
Let $\hat{r}$ denote the normalized radial vector.
Let furthermore $\hat{z}$ be oriented according to the right-hand rule relative to the direction of travel.
Then, the tangential vector is $\hat{t} = \hat{r} \times \hat{z}$.

\subsection{State vectors}

\subsection{Gravity - G}
Gravity is modelled based on the Newtonian formulation, resulting in a force
$$
\vec{G}(r) = -\frac{\mu}{r^2} \hat{r} 
$$
where r is the distance to the center of Earth from the rocket, and $\mu \approx 3.986\cdot10^{14} m^3 s^{−2}$ is the standard gravitational parameter for Earth.

\subsection{Aerodynamic drag - D}
To model aerodynamic drag, it is assumed that the atmosphere moves at a velocity, independent of radius
$$
\vec{v}_{atm} (\vec{r}) = v_{surf} \cdot \hat{t} (\vec{r})
$$
This allows the definiton of the wind-relative velocity
$$
\vec{v}_{atm,rel} (\vec{r}, t) = \vec{v} (t) - \vec{v}_{atm} (\vec{r})
$$
\subsection{Thrust - D}

\end{document}